\documentclass[a4j,twocolumn]{jsarticle}
\usepackage{subfigure}

\def\Vec#1{\mbox{\boldmath $#1$}}
\usepackage[dvipdfmx]{graphicx}

\setlength{\textheight}{275mm}
\headheight 5mm
\topmargin -30mm
\textwidth 185mm
\oddsidemargin -15mm
\evensidemargin -15mm
\pagestyle{empty}
\begin{document}
\title{システム環境構築ソフト}
\author{情報科学科 西谷研究室 2561 森下慎也}
\date{}
\maketitle
\section{はじめに}
本研究で取り扱うChefとは,膨大な数のサーバーの管理, 設定に用いられるシステム環境構築ソフトウェアである.身近な例を挙げると,GREE, Facebook等の巨大システムのサーバー管理において,Chefが使用されている.実際にChefを使用することでサーバ管理におけるミスや手間が大幅に削減することができる.しかし,身近な物の設定がChefによって行われているが,実際にChefのコードに触れる機会は少なく,学び始めると専門用語の理解に苦しめられることがある.本研究ではChefのサポートツールをhikiのプラグインとして開発し,Chefの学習環境を整えることを目的とする.

\section{Chefについて}
Chefは上記でも挙げた通りRuby記述のシステム環境構築ソフトウェアである.本来プログラミング言語は,システム記述には向かない.rubyというプログラミング言語の文法規則をそのまま使ってシステムが記述され,以下に挙げるような利点がある[1-3].
\begin{enumerate}
\item サーバーの環境設定をするにあたって,用意されている手順書を参照して作業を進めていくというのがよく行われる手法であるが,Chefではその手順書をそのままコードとして扱うことができる.これにより手順書に修正が必要となった時のミスを少なくすることができる[1].
\item Chefは「特定のコマンドを使用して,特定のソフトウェアをインストールする」という考え方でなく,「特定コマンドが使用されている状態であり,特定のソフトウェアがインストールされている状態にする」という考え方になっている[2].
\item Chefでは冪等性が重要であると考えられている.冪等性とは何度実行しても同じ実行結果が得られるという性質のことであり,Chefでは冪等性が基本的に守られているため,何度も実行することで同じソフトウェアがインストールされるということは起こらない[3].
\end{enumerate}
\section{Chefサポートツールの概要}
本研究では以下のような機能を実装する.

\subsection{初心者向け}
GUIによってインストールしたいソフトを選び,最終的に選択したソフトを全てインストールできるレシピを出力する.この機能によって全くChefのコードを触ったことの無い人でもコードを作成し,Chefを触ってみることができる.

\subsection{中級者向け}
新規作成したいレシピのインストール法,オプション設定等を選択していくことで最終的にそのレシピの枠組みを出力する.ある程度の枠組みを提示することでChefプログラムの作成及び作成者の理解を手助けする. 

\subsection{発展}
上記で作成した,または,他の手段で入手・作成したコードをデータベースに登録する.1の機能についてのデータベースを学習者で共有する形で追加できるようにする.

\section{結論と今後の課題}
現状上記の仕様によって学習環境の改善が予想できるが細かいところの改善が必要であると思われる.また,サポートツールとしてはここまでで実装を予定している仕様だけではまだ実用段階としては未完成であるため,あらかじめ保存しておいた設定とユーザーの適用前の設定のソフトウェアの差分をインストールするレシピを自動で作成するといったような仕様を更に追加する予定である.

\section{reference:}
「Chef実践入門」,吉羽 龍太郎, 安藤 祐介, 伊藤 直也, 菅井 祐太朗, 並河 祐貴,(技術評論社 2014).
\end{document}
