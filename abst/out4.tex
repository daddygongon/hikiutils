\documentclass[a4j,twocolumn]{jsarticle}
\usepackage{subfigure}

\def\Vec#1{\mbox{\boldmath $#1$}}
\usepackage[dvipdfmx]{graphicx}
\usepackage{url}

\setlength{\textheight}{275mm}
\headheight 5mm
\topmargin -30mm
\textwidth 185mm
\oddsidemargin -15mm
\evensidemargin -15mm
\pagestyle{empty}
\begin{document}
\title{hiki編集作業を容易にするツールの開発}
\author{情報科学科 西谷研究室 3554 山根 亮太}
\date{}
\maketitle
\section{目的}
本研究の目的は,hikiの編集作業をより容易にするためのツールを開発することである.hikiは通常web上で編集を行っているが,GUIとCUIが混在しており,操作に不便な点がある.そこで,編集操作がCUIで完結するために開発をされたのがhikiutilsである.しかし,そのユーザインタフェースには問題がある.現在hikiでコマンドを入力する際,ユーザーは瞬間的にそのコマンドの機能や意味を判断することが難しく迷いが生じてしまうことがある.なので.この部分を開発することによりこの問題を解決したい.

\section{システムについて}
\subsection{hikiについて}
hikiとはwikiエンジンの1つとされており,プログラミング言語Rubyを用いられることで作られたwikiクローンである.
hikiの主な特徴として
\begin{itemize}
\item オリジナルwikiに似たシンプルな書式
\item プラグインによる機能拡張
\item 出力するHTMLを柔軟に変更可能
\item ページにカテゴリ付けできる
\item CSSを使ったテーマ機能
\item 携帯端末可能
\item InterWikiのサポート
\item HikiFarmに対応
\item ページの追加,編集がしやすい
\end{itemize}
等がある\cite{hiki}.

\subsection{hikiutilsについて}
hikiutilsは,hikiの編集作業を容易に行うことができるようにするツール群である.
hikiutilsはCLI(Command Line Interface)で提供されているため,
プログラミング言語Ruby版のコマンドラインのオプション解析を行いコマンドを標準化にするoptparseが利用されている\cite{opt}.
hikiutilsはプログラミング言語Rubyのライブラリであるgemによって提供されている\cite{gem}.
そのため,CLIのoptparseを簡単に扱うためにRubyのライブラリであるgemが用意されている.

\section{手法}
このようなCLIの階層的なコマンド実装を示したcodeとしてTodoというgemアプリケーションがある\cite{PerfectRuby}.
このプログラムではlib/todo/command/options.rbにサブコマンド用の定義があり,入力されたコマンドはilb/todo/command.rbで判断されることによりlib/todo/command/options.rb内で実行される.hikiutilsのプログラムの構造はTodoアプリケーションのプログラムの構造と似たアプリケーションでありhikiの編集を容易にするためのツール群のため,この部分の編集を行うことでコマンド名を書き換える.

\section{現状}
現在Todoアプリケーションのコードを追いかけて,プログラムの構造を理解している.
また,直感的なコマンドとして以下のようなものを考えた.
\begin{verbatim}
   version            show program Version.
   show               show sources
   add                add sources info
   set VAL            set target id
   open FILE          open file
   list FILE          list files
   update FILE        update file
   rsync              rsync files
   read db FILE       read database file
   display FILE       display converted hikifile
   check db FILE      check database file
   remove FILE        remove file
   move file1 file2   move file1,file2
   euc FILE           translate file to euc
\end{verbatim}
上記のコマンドはgit風のコマンド名に変更してみた.
今後の課題としては,実際にhikiutils内でそのコマンドの編集を行って実装し,そのコマンドが分かりやすいものであるかどうかを試してもらう必要がある.

\begin{thebibliography}{9}
\bibitem{hiki}「hiki」,hikiwiki.org/ja/about.html,2016/9/2 アクセス.
\bibitem{opt}「optparseの使い方」,\url{dharry.hatenablog.com/entry/20081008/1223490673},2016/9/2 アクセス.
\bibitem{gem}「Rubygems-Wikipedia」,\url{ja.wikipedia.org/wiki/RubyGems},2016/9/2 アクセス.
\bibitem{PerfectRuby}「パーフェクトRuby」,Rubyサポーターズ(技術評論社 2013).
\end{thebibliography}
\end{document}
