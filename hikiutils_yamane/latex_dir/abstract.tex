\section{概要}
研究室内の内部文書,あるいは外部への宣伝資料,さらにwikipediaのように重要な研究成果の発信などに西谷研ではhiki systemを利用している.
これは初心者にも覚えやすい直感的な操作であるが,慣れてくるとテキスト編集や画面更新にいちいちweb画面へ移行せねばならず,編集の思考が停止する.
そこで,編集操作がCUIで完結させるためにテキスト編集に優れたeditorとの連携や,terminal上のshell commandと連携しやすいhikiutilsが開発された.
しかし,そのユーザーインターフェースにはコマンドが直感的でないという問題がある.
そこで,本研究ではコマンドラインツール作成ライブラリを変更することでコマンドを実装し直し直感的なコマンドにすることを目的とした.
optparseで作成されているhikiutilsをthorで作成し,そして2つのコマンドラインツール作成ライブラリで作成されたhikiutilsを比較する.
研究結果は,thorのほうがコマンドを簡単に定義することができ,またコードも短くできた.

