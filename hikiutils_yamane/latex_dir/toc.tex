\section{1:概要}
\section{2:序論}
\section{3:方法}
\subsection{3.1:optparseとthorの比較}
\subsubsection{3.1.1:optparse}
\subsubsection{3.1.2:Thor}
\subsection{3.2:既存のhikiutilsのコマンド解説}
\subsubsection{3.2.1:コマンドの登録と実行メソッド}
\subsubsection{3.2.2:CLIの実行プロセス}
\subsubsection{3.2.3:コード}
\section{4:結果}
\subsection{4.1:コマンドの命名原則}
\subsubsection{4.1.1:hikiutilsの想定利用形態}
\subsubsection{4.1.2:コメンド名と振る舞いの詳細}
\subsection{4.2:Thorによる実装}
\subsubsection{4.2.1:クラス初期化}
\subsubsection{4.2.2:コマンド定義}
\subsubsection{4.2.3:CLIの実行プロセス}
\subsubsection{4.2.4:optparseとの全体的な比較}
