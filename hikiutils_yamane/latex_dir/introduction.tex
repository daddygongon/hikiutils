\section{序論}
\subsection{目的}
本研究ではhikiの編集作業をより容易にするためのツールの開発を行った. 
hikiは通常web上で編集を行っているが,GUIとCUIが混在しており,操作に不便な点がある. 
そこで,編集操作が CUI で完結するために開発をされたのが hikiutils である. 
しかし,そのユーザインタフェースにはコマンドが直感的でないという問題点がある. 
そこで,Thorというコマンドラインツール作成ライブラリを用いる.
optparseというコマンドライン解析ライブラリを使用しているhikiutilsを新たなコマンドライン解析ライブラリを使用することコマンドを書き換え,より直感的なコマンドにすることが可能である.

\subsection{既存システムの背景}
\subsubsection{hiki}
hikiとはプログラミング言語Rubyを用いられることで作られたwikiクローンの1つである.
hikiの主な特徴として
\begin{itemize}
\item オリジナルwikiに似たシンプルな書式
\item プラグインによる機能拡張
\item 出力するHTMLを柔軟に変更可能
\item ページにカテゴリ付けできる
\item CSSを使ったテーマ機能
\item 携帯端末可能
\item InterWikiのサポート
\item HikiFarmに対応
\item ページの追加,編集がしやすい
\end{itemize}
等がある[1].

\subsubsection{hikiutils}
hikiutils は hiki の編集作業を容易に行うことができるよう にするツール群であり,プログラミング言語 Ruby のライブ ラリである gem フォーマットに従って提供されている[2]. 
hikiutils は CLI(Command Line Interface) で操作するため, オプション解析をおこなう必要がある. 
gem には,この用途 に適合したライブラリがいくつも提供されている[3]. 
この中 で,あまり利用頻度は高くないが古くから開発され,使用例 が広く紹介されている optparse を利用している.

