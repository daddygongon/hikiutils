\section{序論}
\subsection{目的}
\subsection{既存システムの背景}
\subsubsection{hiki}
hikiとはプログラミング言語Rubyを用いられることで作られたwikiクローンの1つである.
hikiの主な特徴として

\begin{itemize}
\item オリジナルwikiに似たシンプルな書式
\item プラグインによる機能拡張
\item 出力するHTMLを柔軟に変更可能
\item ページにカテゴリ付けできる
\item CSSを使ったテーマ機能
\item 携帯端末可能
\item InterWikiのサポート
\item HikiFarmに対応
\item ページの追加,編集がしやすい
\end{itemize}
等がある

\subsubsection{hikiutils}
hikiutils は hiki の編集作業を容易に行うことができるよう にするツール群であり,プログラミング言語 Ruby のライブ ラリである gem フォーマットに従って提供されている. 
hikiutils は CLI(Command Line Interface) で操作するため, オプション解析をおこなう必要がある. 
gem には,この用途 に適合したライブラリがいくつも提供されている. 
この中 で,あまり利用頻度は高くないが古くから開発され,使用例 が広く紹介されている optparse を利用している.

