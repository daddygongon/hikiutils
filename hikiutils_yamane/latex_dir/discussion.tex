\section{optparseからThorへの移行}
Thorとoptparseでのコードの違いは以上のとおりであるが,コードからもThorのほうが短くなっていることが分かる.
しかし,Thorの問題点はメソッド名がコマンドとなるため,1つしか定義できないことである.
これを解決するためにmapを用い,複数のコマンドを定義できるようにした.
一方,optparseでは別のコマンドを定義するにはfizzbuzzのoptparseのコードのようにコマンドの解析を行う必要がある.
つまり,optparseでのコマンド定義はThorより複雑で記述が長くなるということである.
それに対してThorのほうが全体的にもコードが短くなり,コマンドの定義も簡単に行うことができる.
また,実行手順も分かりやすくコードが読みやすいため書き換えもすぐ行うことができるので,より直感的なコマンドを実装することも可能となった.

