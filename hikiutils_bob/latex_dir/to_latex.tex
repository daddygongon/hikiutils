\subsection{実際の最終形態(卒論=pdf)への変換}
hiki+keynoteで卒論の内容が出来てきたら,それを実際の最終形態つまり
卒論のpdfへ変換する必要が出てきます.

ここで問題が発生します.hikiよりもlatexの方が機能が豊富なこと.
つまり,簡易に書くにはhikiなどのmark downで書いていくのがいいんですが,
文書として完成度高めるには,latexで細かい設定を調整する必要がどうしても出てきます.

具体的には,
\begin{enumerate}
\item 図の配置を調整するwrapの数値調整
\item 参照文献の記述
\item リスト,引用の体裁
\item 章立て階層構造
\end{enumerate}
などが問題になるところです.

これはhikiではどうしようもありません.一つの手はもう少し高機能なmark up言語
例えばasciidocなどに変更することですが,これはどこまでいっても終わりのない
方向のようで,結局はlatexで書いているのと同じになる可能性があります.

我々は違う戦略をとります.それは,
\begin{quote}\begin{verbatim}
latexをベースにして,hikiを生成する
\end{verbatim}\end{quote}
と言う手です.

DRY(Don't Repeat Yourself)原則さへ守ればいいのですから,ある段階まで
すすめばhikiではなくlatexを原本にするのです.そのための変換器latex2hikiと
その派生rake環境が用意されています.そちらに進んでください.

ただ,その前にhikiでできることをとことん突き詰めておきます.
これは,単純な文書,たとえばabstractとか中間発表のhandout程度では十分に役に立つコマンド群です.

\subsubsection{install}\begin{quote}\begin{verbatim}
hiki -i
\end{verbatim}\end{quote}
でinstallされています.新たに使うコマンド群は次の通りです.
\begin{lstlisting}[style=]
rake change_wrap     # change latex figures to wrap format
rake latex           # latex conversion FILE1
rake latex_base      # latex conversion FILE1(hiki) to FILE2(latex)
rake latex_wrap      # latex conversion FILE1 with wrap format
\end{lstlisting}
