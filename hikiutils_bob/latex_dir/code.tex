\section{Rakefile開発に関するメモ}
\subsection{ファイル名とdirectoryの対応表}\begin{quote}\begin{verbatim}

\end{verbatim}\end{quote}
\begin{table}[htbp]\begin{center}
\caption{}
\begin{tabular}{llll}
\hline
執筆directory:test\_bob  &hikiシステム  &latex\_dir  \\ \hline
test\_bob/test\_bob.hiki  &test\_bob  &test\_bob/latex\_dir/test\_bob.tex  \\
test\_bob/hiki\_dir/test\_bob.hiki  &test\_bob  &test\_bob/latex\_dir/test\_bob.tex  \\
test\_bob/figs/test0.png  &test\_bob\_figs/test0.png  \\
\hline
\end{tabular}
\label{default}
\end{center}\end{table}
%for inserting separate lines, use \hline, \cline{2-3} etc.

\subsection{日本語のcode listings}
日本語のjlistingの使用法が判明しました.
listingsでは日本語表示がちゃんとなされません.そこで,
\begin{quote}\begin{verbatim}
\usepackage{listings,jlisting}
\end{verbatim}\end{quote}
とします.これで,日本語が含まれたcodeも綺麗に表示してくれます.

\subsection{システムの概要}
卒論編集システム開発時のメモです.図\ref{fig:SystemOverview}に卒論編集システムの概観を示しています.

hikiシステムとの同期は,hikiutilsのhikiが下請けしている.
一方,latex\_dirへの出力はlatex2hikiが引き受けている.
フォーマットをいじるときには,基本的に
\begin{description}
\item[hikiへ] hikiがやるので,そのまえにtmp.txtへ写して置換

\item[texへ] latex2hikiからの出力(tmp.txt)を処理してtexへ

\end{description}
で行っている,あるいはおこなう.

\subsection{参照機能の実装}
Latexの参照機能を通すための変換を行っています.そのための記述の仕方は\ref{fig:CiteRefSystems}を参照してください.実際のcodeはRakefile内で

\begin{itemize}
\item hikiはrake syncした時にtmpで変換させてそれを転送
\item latexはhiki2latexで出てきた結果を変換
\end{itemize}
しています.hiki2latex, latex2hikiへの実装も考えたのですが...

attach\_anchor, titleとかは変換しているので,そちらで実装する方がいいかもしれません.
どこまでを汎用ライブラリでさせて,どこからを個別のRafefileに入れるかの判断基準を
これから明確化していく必要があるでしょう.

\subsection{rake mk\_toc}
latexが作成するtableofcontentsの実態であるtocファイルからhiki用のtoc.hikiを作成する.

