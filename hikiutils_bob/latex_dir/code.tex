\section{latexからhikiへの変換}
\subsection{rake mk\_toc}
latexが作成するtableofcontentsの実態であるtocファイルからhiki用のtoc.hikiを作成する.

\subsection{listings}\begin{lstlisting}[style=customTeX,basicstyle={\scriptsize\ttfamily}]
\lstset{
  basicstyle={\ttfamily\scriptsize},
  identifierstyle={\scriptsize},
  commentstyle={\scriptsize\itshape\color{red}},
  keywordstyle={\scriptsize\bfseries\color{cyan}},
  ndkeywordstyle={\scriptsize},
  stringstyle={\scriptsize\color{blue}},
  frame={tb},
  breaklines=true,
  numbers=left,
  numberstyle={\footnotesize},
  stepnumber=1,
  numbersep=1zw,
  xrightmargin=0zw,
  xleftmargin=3zw,
  lineskip=-0.5ex
}
\end{lstlisting}
\subsection{日本語のcode listings}
日本語のjlistingの挙動がようやく判明しました.listingsでは日本語表示が
ちゃんとなされません.そこで,
\begin{quote}\begin{verbatim}
\usepackage{listings,jlisting}
\end{verbatim}\end{quote}
としています.これで,日本語が含まれたcodeも綺麗に表示してくれます.

\begin{itemize}
\item \verb|{{cite(listings1)}}|
\item \verb|{{cite(listings2)}}|
\end{itemize}
\section{Reference:}\begin{description}
\item[listings1] \verb|基本的な使い方(http://d.hatena.ne.jp/mallowlabs/20061226/1167137637)|

\item[listings2] \verb|listingsの定義の仕方(http://www.ipc.akita-nct.ac.jp/~yamamoto/comp/latex/make_doc/source/source.html)|
\end{description}
