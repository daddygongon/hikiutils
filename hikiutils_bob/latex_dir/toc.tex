\section{1:イントロダクション}
\section{2:hikiとの同期}
\subsection{2.1:hikiutilsのinstall}
\subsection{2.2:個別ディレクトリーの構成}
\subsection{2.3:システムの概要}
\subsection{2.4:一般的な執筆手順}
\subsection{2.5:rakeが用意しているタスク}
\subsubsection{2.5.1:rake sync}
\subsubsection{2.5.2:rake force\_sync}
\subsubsection{2.5.3:rake chenv}
\subsubsection{2.5.4:rake convert VAL DIR}
\subsubsection{2.5.5:rake increment}
\subsubsection{2.5.6:rake number}
\subsubsection{2.5.7:rake touch}
\subsubsection{2.5.8:rake github}
\subsubsection{2.5.9:rake reset\_hiki}
\subsection{2.6:githubによる同期}
\subsection{2.7:hikiutil関連のヘルプ}
\subsubsection{2.7.1:hikiで卒論を書くときの初期化と掟}
\subsubsection{2.7.2:error対応}
\subsubsection{2.7.3:図表:すべての図表をkeynoteにまとめる,タイトルを分かりやすく書く}
\section{3:実際の最終形態(卒論=pdf)への変換}
\subsection{3.1:install}
\subsection{3.2:rake latex(個別ファイルの変換)}
\subsubsection{3.2.1:注意}
\subsection{3.3:rake latex\_all手順(ディレクトリー内の一括変換)}
\subsubsection{3.3.1:下準備}
\subsection{3.4:補助コマンドの解説}
\subsubsection{3.4.1:rake reset\_latex\_dir(latex\_dirのゴミ掃除)}
\subsubsection{3.4.2:wrap関係}
\section{4:hikiへのlatexからの変換}
\subsection{4.1:rake mk\_toc}
