
\section{実際の最終形態(卒論=pdf)への変換}
hiki+keynoteで卒論の内容ができたら,それを卒論の最終形態つまりpdfへ変換する必要が出てきます.

ここで問題が発生します.hikiよりもlatexの方が機能が豊富なこと.
つまり,簡易に書くにはhikiなどのmark downで書いていくのがいいんですが,
文書として完成度を高めるには,latexで細かい設定を調整する必要がどうしても出てきます.

具体的には,

\begin{enumerate}
\item 図の配置を調整するwrapの数値調整
\item 参照文献の記述
\item リスト,引用の体裁
\item 章立て階層構造
\end{enumerate}
などが問題になるところです.

これはhikiではどうしようもありません.一つの手はもう少し高機能なmark up言語
例えばasciidocなどに変更することですが,これはどこまでいっても終わりのない
方向のようで,結局はlatexで書いているのと同じになる可能性があります.

我々は違う戦略をとります.それは,
\begin{quote}\begin{verbatim}
latexをベースにして,hikiを生成する
\end{verbatim}\end{quote}
と言う手です.

DRY(Don't Repeat Yourself)原則さえ守れば,文書管理はいいのですから,
ある段階まですすめばhikiではなくlatexを原本にするのです.
そのための変換器latex2hikiとその派生rake環境が用意されています.
そちらはこちらで卒業後に変換します.参照文献はlatexに移してから
修正してください.

ここでは,hikiでできることをとことん突き詰めておきます.

\subsection{install}\begin{quote}\begin{verbatim}
hiki -i
\end{verbatim}\end{quote}
でinstallされています.新たに使うコマンド群は次の通りです.
\begin{lstlisting}[style=]
rake change_wrap     # change latex figures to wrap format
rake latex           # latex conversion FILE1
rake latex_all       # latex conversion whole hiki files in the current dir
rake latex_base      # latex conversion FILE1(hiki) to FILE2(latex)
rake latex_wrap      # latex conversion FILE1 with wrap format
\end{lstlisting}
\subsection{rake latex手順(個別ファイルの変換)}\begin{quote}\begin{verbatim}
rake latex sync.hiki
\end{verbatim}\end{quote}
とすると,
\begin{quote}\begin{verbatim}
latex_dir/sync.tex
\end{verbatim}\end{quote}
にlatex変換後のファイルが生成します.これを,TeXShopでcommand\_t変換します.

\begin{itemize}
\item \verb|{{attach_anchor(sync.pdf,hikiutils_bob)}}|
\end{itemize}
うまくいかないときは,terminalで
\begin{quote}\begin{verbatim}
platex sync.tex
dvipdfmx sync.dvi
\end{verbatim}\end{quote}
を試してみてください.

\subsubsection{注意}
hikiの初めの部分は,
\begin{quote}\begin{verbatim}

\end{verbatim}\end{quote}
とすると,
\begin{lstlisting}[style=]
\begin{document}
\title{hikiutils -iによる卒論作成システム}
\author{Shigeto R. Nishitani}
\date{ Kwansei Gakuen Univ., 2017/1}
\maketitle
\end{lstlisting}
と変換してくれます.

必要な図は,figsから自動で取るようになっていますが,
サイズや解像度が問題のときは手動で調整してください.

\subsection{rake latex\_all手順(ディレクトリー内の一括変換)}
pwdのdirectoryと同名のbasenameに.hikiの拡張子がついたファイルが用意されていて,
\begin{quote}\begin{verbatim}
rake latex_all
\end{verbatim}\end{quote}
をおこなうと自動で全部を一体にまとめた文書へのlatex変換が出来上がります.
たとえば,hikiutils\_bob.hikiの記述を
\begin{quote}\begin{verbatim}
bob% cat hikiutils_bob.hiki 
!hikiutilsを用いた卒業論文作成

! [[hikiutils_bob_sync]]
! [[hikiutils_bob_latex_all]]

\end{verbatim}\end{quote}
とすると,sync.tex, latex\_all.texがlatex\_dir内に変換されます.
さらに,開いているhikiutils\_bob.texをTeXShopで変換してみてください.

うまくできないときは\verb|ここ(http://qiita.com/hideaki_polisci/items/3afd204449c6cdd995c9)|を参照して自力で入れてみてください.だめならdonkey.

\begin{itemize}
\item \verb|{{attach_anchor(hikiutils_bob.pdf,hikiutils_bob)}}|
\end{itemize}
というようになります.

\subsubsection{下準備}
latex\_dir内に幾つかのtex雛形を入れておく必要があります.
\begin{quote}\begin{verbatim}
hiki -i
\end{verbatim}\end{quote}
とhikiutils環境を再度初期化すると,自動でインストールする設定です.
なかったら手動で作ってください.

\paragraph{head.tex}
題目,学生番号,氏名を変更する.年月をチェック.
\begin{lstlisting}[style=]
bob% cat head.tex
\title{卒業論文\\
\vspace{4cm} hikiutilsを用いた\\卒業論文作成}
\author{ 関西学院大学 理工学部 情報科学科\\\\1234 西谷滋人}
\date{\vspace{3cm} 2017年  3月\\
\vspace{3cm} 指導教員  西谷 滋人 教授}
\maketitle

\end{lstlisting}
\paragraph{pre.tex}
あまり変更しないほうがいいが,いずれいじる.例えば,フォントポイント数を10から12ptに変える,
\pagestyle{empty}を外すなど.
\begin{lstlisting}[style=]
bob% cat pre.tex
\documentclass[10pt,a4j]{article}

\def\Vec#1{\mbox{\boldmath $#1$}}
\usepackage[dvipdfmx]{graphicx}

\setlength{\textheight}{275mm}
\headheight 5mm
\topmargin -20mm
\textwidth 160mm
\textheight 250mm
\oddsidemargin -0mm
\evensidemargin -5mm

\pagestyle{empty}
\makeatletter
  \def\@maketitle{%
  \newpage\null
  \vskip 2em%
  \begin{center}%
  \let\footnote\thanks
    {\large\bf \@title \par}%
    \vskip 1.5em%
    {\large\bf \@author \par}%
    \vskip 1.5em%
    {\small \@date}%
  \end{center}%
}
\makeatother

%\documentclass[10pt, a4j]{article}
%%\usepackage{citesort}
\usepackage{amssymb}
\usepackage[dvipdfmx]{graphicx}% 図を入れるときに使用
\usepackage{wrapfig}% 図の周りに本文を流し込みたいときに使用
\usepackage{subfigure}
\usepackage{here}
\end{lstlisting}
